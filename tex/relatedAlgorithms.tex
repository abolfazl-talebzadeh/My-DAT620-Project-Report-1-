Basically, in this project multiple algorithms has been used which can be divided into two groups, the first one is the algorithms which has been used to infer some result coherence and logic to make decisions on business related acts and the second one is the algorithms which has been implemented to apply sentiment analysis on customer reviews. the first group are those which has been implemented on Hadoop and consists of 7 algorithms:
\subsection{Number of Reviews Each Year} this algorithm has been implemented using one mapper and one reducer functions. in the mapper function first, the whole line is read and sent as line into the mapper, the white spaces which is located in the beginning and at the end of the line will be dropped by using strip  python function, the stripped string then is separated and saved as a list; the 14th element of the list which is the date the review is made is selected and again separated into day, month and the year. To figure out that the date is not null or doesn't have an invalid value, within a condition, specifies if the length of separated value is equal to 3 (representing day, month and year), then the values from the first position to position number seven of the original non-separated value (to have year and month together e.g., 2002-12 for December of 2002) for date will be returned as the key and "1" as the value.

In reduce function, the list of values will be added through python function sum and is return as the value along side the key which has remained untouched from the mapper function. In code~\ref{lst:revYearCount} you see the implementation of this algorithm.




\renewcommand{\lstlistingname}{Code}
\lstset{style=mystyle}
\begin{lstlisting}[language=Python, caption={Number of Reviews Each Year}, label={lst:revYearCount}, mathescape = true, breaklines=true]
from mrjob.job import MRJob

class MRCountSum(MRJob):

    def mapper(self, _, line):
        line = line.strip()  
        lineList = line.split("\t")
        date = lineList[14]
        if len(date.split("-"))==3:
            yield date[:7],1
        yield "", 0
    def combiner(self, key, values):
        yield key, sum(values)
        
    def reducer(self, key, values):
        yield key, sum(values)

if __name__ == '__main__':
    MRCountSum.run()
\end{lstlisting}

\rule{200 pt}{0.5 pt} 


\subsection{Number of Reviews Each Customer Has Made}
The purpose of writing this algorithm has been discussed so in this part we go through the implementation and algorithm. This algorithm has been implemented using one mapper function and one reducer function. the mapper function receives the each line through line argument in the function. The line will then divided into 15 fields and stored in a list. Customer identification number is fetched from the list and stored in a variable and is returned as the key and number "1" is returned as the value to the reducer. in the reducer the unique values for each customer ID is received and the summation of number of "1" values is calculated and return to the output with the key. the code for this algorithm can be found on code~\ref{lst:CostumerCount} 

\rule{200 pt}{0.5 pt} 

\renewcommand{\lstlistingname}{Code}
\lstset{style=mystyle}
\begin{lstlisting}[language=Python, caption={Number of Reviews Each Customer Has Made}, label={lst:CostumerCount}, mathescape = true, breaklines=true]
from mrjob.job import MRJob

class MRCountSum(MRJob):

    def mapper(self, _, line):
        line = line.strip() # remove leading and trailing whitespace
        lineList = line.split("\t")
        customer = lineList[1]
        yield customer, 1
    def combiner(self, key, values):
        yield key, sum(values)
        
    def reducer(self, key, values):
        yield key, sum(values)

if __name__ == '__main__':
    MRCountSum.run()
\end{lstlisting}

\rule{200 pt}{0.5 pt} 

\subsection{Number of Reviews for Each Month}
This algorithm includes two mapper functions and two reducer functions. At the top of everything in the class    a function named steps has defined. This takes care of the order of running multiple mappers and reducers.

The first mapper function which is called "mapper\_count", receives the line as a single consistent piece of strings containing 15 values which has been put together using tab character. At the next step the excess spaces at the beginning and the end of the string will be removed using strip function. Then it is divided into 15 parts which represent the values for each field and is stored in a list. In this mapper function, to avoid receiving the header as a line of values, the value is compared to "review\_date " and then the values from the first position to position number seven of the original non-separated value (to have year and month together e.g., 2002-12 for December of 2002) for date will be returned as the key and "1" as the value. if it was equal to the corresponding header title, value null and 0 will be returned.

In the first reducer function, the key which is the truncated date remains untouched and gets returned and the summation of the list of "1"s also is returned to the next mapper as the value.

In the Second mapper function, the received key is divided by the dividing factor "-". and the first element which is the year will be returned as the key and the value which is the total number of reviews for that month of the year will be returned as the value. In the last reducer, the value of year is returned as the key and the values for each month is stored in a list and the maximum value and the position of the maximum value will be returned as the hottest and the year and the minimum value and the position of the minimum value in the list will be returned as the coldest.In code~\ref{lst:reviewsPerMonth} you can find the corresponding code for this algorithm.

\rule{200 pt}{0.5 pt} 

\renewcommand{\lstlistingname}{Code}
\lstset{style=mystyle}
\begin{lstlisting}[language=Python, caption={Number of Reviews for Each Month}, label={lst:reviewsPerMonth}, mathescape = true, breaklines=true]

from mrjob.job import MRJob
from mrjob.step import MRStep

class MRCountSumAvg(MRJob):
    def steps(self):
        return [
            MRStep(mapper=self.mapper_count,
                   combiner=self.combiner_count,
                   reducer=self.reducer_count),
            MRStep(mapper=self.mapper_avg,
                   reducer=self.reducer_avg)
            ]

    def mapper_count(self, _, line):
        line = line.strip()  
        lineList = line.split("\t")
        date = lineList[14]
       # dataList = date.split("-")
        if date.strip() !="review_date":#len(date)==10:
            yield date[:7],1
        yield "", 0

    def combiner_count(self, key, values):
        yield key, sum(values)
        
    def reducer_count(self, key, values):
        yield key, sum(values)
        
    def mapper_avg(self, key, value):
        sdate = key.split("-")
        yield sdate[0], value

    def reducer_avg(self, key, values):
        values_list = list(values)
        if len(values_list)<1:
            yield "No values", 0
        maxOut = "Hotest month on "+str(key)
        minOut = "Coldest month on "+str(key)
        minList = [values_list.index(min(values_list))+1,min(values_list)]
        maxList = [values_list.index(max(values_list))+1,max(values_list)]
        yield maxOut, maxList
        yield minOut, minList

if __name__ == '__main__':
    MRCountSumAvg.run()

\end{lstlisting}

\rule{200 pt}{0.5 pt} 


\subsection{Number of Reviews for Each Product}
The reason for developing this algorithm has already been stated in \ref{sec:background}, therefore we'll move on to the implementation and algorithm in this section. One mapper function and one reducer function were used to implement this technique. Each line is received by the mapper function through the line parameter. After that, the line will be broken down into 15 fields and kept in a list. The product identification number is retrieved from the list, saved in a variable, and returned to the reducer as the key, with the value "1" as the value. The unique values for each product ID are received in the reducer, and the total of the number of "1" values is computed before returning to the output with the key. In code~\ref{lst:reviewsProduct} you can find the corresponding code for this algorithm.


\rule{200 pt}{0.5 pt} 

\renewcommand{\lstlistingname}{Code}
\lstset{style=mystyle}
\begin{lstlisting}[language=Python, caption={Number of Reviews for each Product}, label={lst:reviewsProduct}, mathescape = true, breaklines=true]

from mrjob.job import MRJob

class MRCountSum(MRJob):

    def mapper(self, _, line):
        line = line.strip()  
        lineList = line.split("\t")
        productTitle = lineList[5].strip()
        yield productTitle, 1
    def combiner(self, key, values):
        yield key, sum(values)
        
    def reducer(self, key, values):
        yield key, sum(values)

if __name__ == '__main__':
    MRCountSum.run()

\end{lstlisting}

\rule{200 pt}{0.5 pt} 

\subsection{Average Number of Reviews per Month During Each Year} 
Two mapper functions and two reducer functions are included in this algorithm. A function called steps has been defined at the top of the class. This ensures that many mappers and reducers are performed in the correct order.

The first mapper function, "mapper count," takes the line as a single consistent string comprising 15 numbers that has been joined together with the tab character. The superfluous spaces at the beginning and end of the string will be eliminated using the strip function in the following step. It is then separated into 15 pieces, each of which represents a field's value, and kept in a list.To avoid receiving the header as a line of values, the value is compared to "review date," and the values from the first to the seventh positions of the original non-separated value (to have year and month together, e.g., 2002-12 for December of 2002) for date are returned as the key and "1" as the value. Value null and 0 will be returned if it was equivalent to the relevant header title.

The key, which is the shortened date, is returned unaltered in the first reducer function, and the sum of the list of "1"s is likewise sent to the next mapper as the value.The received key is split by the dividing factor "-" in the Second mapper function. The year will be provided as the key, and the value, which will be the total number of reviews for that month of the year, will be returned as the value. The value of the year is returned as the key, and the values for each month are saved in a list, the average number of the review per month is calculated by dividing the sum of the values by the number of them. T the maximum and the minimum are also calculated by using min and max functions and returned separately with the year. You can refer to the source code of the implementation of this algorithm in code~\ref{lst:avgRev}


\rule{200 pt}{0.5 pt} 

\renewcommand{\lstlistingname}{Code}
\lstset{style=mystyle}
\begin{lstlisting}[language=Python, caption={Average Number of Reviews per Month During Each Year}, label={lst:avgRev}, mathescape = true, breaklines=true]


from mrjob.job import MRJob
from mrjob.step import MRStep

class MRCountSumAvg(MRJob):
    def steps(self):
        return [
            MRStep(mapper=self.mapper_count,
                   combiner=self.combiner_count,
                   reducer=self.reducer_count),
            MRStep(mapper=self.mapper_avg,
                   reducer=self.reducer_avg)
            ]

    def mapper_count(self, _, line):
        line = line.strip()  
        lineList = line.split("\t")
        date = lineList[14]
       # dataList = date.split("-")
        if date.strip() !="review_date":#len(date)==10:
            yield date[:7],1
        yield "", 0

    def combiner_count(self, key, values):
        yield key, sum(values)
        
    def reducer_count(self, key, values):
        yield key, sum(values)
        
    def mapper_avg(self, key, value):
        sdate = key.split("-")
        yield sdate[0], value

    def reducer_avg(self, key, values):
        values_list = list(values)
        avgOut = "Average reviews per month on "+str(key)
        maxOut = "Maximum reviews per month on "+str(key)
        minOut = "Minimum reviews per month on "+str(key)
        yield avgOut, sum(values_list)/len(values_list)
        yield maxOut, max(values_list)
        yield minOut, min(values_list)

if __name__ == '__main__':
    MRCountSumAvg.run()
\end{lstlisting}

\rule{200 pt}{0.5 pt} 

