The related studies which is done are mostly on sentiment analysis and its usage on developing policies for business entrepreneurs. a look through the papers leads us mostly to a list of some works done on different techniques for implementing sentiment analysis. the items below gives us an idea of what the related studies are about:
\begin{itemize}
    \item Electronic market analytic based on customer reviews: Classic challenges in e-marketing such as pricing, brand positioning and new product development are rooted in product analysis based on users’ feedback \cite{2008}.
    \item Sentiment analysis using the product review data: Sentiment analysis also known as the emotion AI, uses natural language processing to analyze the emotions from the extracted opinions\cite{sari_measuring_2018}.
    \item Opinion mining and sentiment analysis on online customer review: The process of finding user opinion about the topic or product is called as opinion mining. The goal of Opinion Mining and Sentiment Analysis is to make computer able to recognize and express emotion\cite{kumar_opinion_2016}.
    \item Classification of customer review based on the sentiment analysis: It is demonstrated that if customers' reviews are classified as negative or positive, classifications is correct with a probability of more than 90\% \cite{fang_sentiment_2015}.
    \item The analysis and prediction of customer review rating using opinion mining in which customers commented as open opinion using probability's classifier model\cite{songpan_analysis_2017-1}.
    \item Measuring e-commerce service quality from online customer review using sentiment analysis, in which more specifically the Naive Bayes classification is the most common methodology for being highly accurate and supporting big data processing.
\end{itemize}
